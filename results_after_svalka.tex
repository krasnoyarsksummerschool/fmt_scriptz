% This is a generated file, do not edit
\documentclass[12pt]{article}
\usepackage[a4paper,top=1cm,bottom=1cm]{geometry}

\usepackage{graphicx}
\usepackage{color}

\usepackage[utf8]{inputenc}
\usepackage[T2A]{fontenc}
\usepackage[russian]{babel}

\begin{document}

\pagestyle{empty}

\begin{center}
\begin{tabular}{cc}
\includegraphics[scale=0.48]{../klshlogo.pdf} &
\raisebox{0.5cm}{
  \begin{tabular}{c}
    {\Large\bf КРАСНОЯРСКАЯ ЛЕТНЯЯ ШКОЛА}\\
    \ \\
    {\large\bf Сезон 2010 года}\\
   \end{tabular}
}
\end{tabular}
\ \\
\ \\
\ \\
{\Large\bf Результаты физико--математического турнира}
\ \\
\end{center}

Команды расположены в таблице в порядке убывания итогового рейтинга. 
{\em Рейтингом} команды называется количество команд, выступивших
менее успешно. Итоговый рейтинг есть сумма рейтингов в основном туре и
свалке. В случае, когда две команды имеют одинаковую сумму рейтингов в
основном туре и свалке, более успешной считается команда, чей рейтинг
в основном туре выше.

Команды, занимающие \textcolor{red}{первое и второе место} в итоговом
рейтинге, встречаются в суперфинале за абсолютное первое
место. Команды, занимающие \textcolor{green}{третье и четвёртое
место}, встречаются за абсолютное третье место.

\vspace*{1cm}

\begin{center}
    \begin{tabular}{|c|c|c|c|}

      \hline
      {\bf Команда} & {\bf Основной тур} &
      {\bf Финальная свалка} & {\bf Итоговый рейтинг} \\

      \hline
      \hline
\textcolor{red}{$\rho$} & 
\textcolor{red}{17} & 
\textcolor{red}{15} & 
\textcolor{red}{32} \\ 
\hline
\textcolor{red}{$\iota$} & 
\textcolor{red}{14} & 
\textcolor{red}{17} & 
\textcolor{red}{31} \\ 
\hline
\textcolor{green}{$\lambda$} & 
\textcolor{green}{16} & 
\textcolor{green}{14} & 
\textcolor{green}{30} \\ 
\hline
\textcolor{green}{$\delta$} & 
\textcolor{green}{14} & 
\textcolor{green}{16} & 
\textcolor{green}{30} \\ 
\hline
\textcolor{black}{$\sigma$} & 
\textcolor{black}{13} & 
\textcolor{black}{12} & 
\textcolor{black}{25} \\ 
\hline
\textcolor{black}{$\chi$} & 
\textcolor{black}{11} & 
\textcolor{black}{13} & 
\textcolor{black}{24} \\ 
\hline
\textcolor{black}{$\beta$} & 
\textcolor{black}{8} & 
\textcolor{black}{11} & 
\textcolor{black}{19} \\ 
\hline
\textcolor{black}{$\xi$} & 
\textcolor{black}{8} & 
\textcolor{black}{8} & 
\textcolor{black}{16} \\ 
\hline
\textcolor{black}{$\alpha$} & 
\textcolor{black}{11} & 
\textcolor{black}{4} & 
\textcolor{black}{15} \\ 
\hline
\textcolor{black}{$\gamma$} & 
\textcolor{black}{7} & 
\textcolor{black}{8} & 
\textcolor{black}{15} \\ 
\hline
\textcolor{black}{$\varphi$} & 
\textcolor{black}{8} & 
\textcolor{black}{6} & 
\textcolor{black}{14} \\ 
\hline
\textcolor{black}{$\omega$} & 
\textcolor{black}{6} & 
\textcolor{black}{8} & 
\textcolor{black}{14} \\ 
\hline
\textcolor{black}{$\ta$} & 
\textcolor{black}{2} & 
\textcolor{black}{6} & 
\textcolor{black}{8} \\ 
\hline
\textcolor{black}{$\psi$} & 
\textcolor{black}{4} & 
\textcolor{black}{2} & 
\textcolor{black}{6} \\ 
\hline
\textcolor{black}{$o$} & 
\textcolor{black}{3} & 
\textcolor{black}{3} & 
\textcolor{black}{6} \\ 
\hline
\textcolor{black}{$\theta$} & 
\textcolor{black}{5} & 
\textcolor{black}{0} & 
\textcolor{black}{5} \\ 
\hline
\textcolor{black}{$\kappa$} & 
\textcolor{black}{0} & 
\textcolor{black}{4} & 
\textcolor{black}{4} \\ 
\hline
\textcolor{black}{$\varepsilon$} & 
\textcolor{black}{1} & 
\textcolor{black}{0} & 
\textcolor{black}{1} \\ 
\hline
\hline

    \end{tabular}

  \end{center}

\end{document}
